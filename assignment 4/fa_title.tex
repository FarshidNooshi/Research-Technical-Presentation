%% -!TEX root = AUTthesis.tex
% در این فایل، عنوان پایان‌نامه، مشخصات خود، متن تقدیمی‌، ستایش، سپاس‌گزاری و چکیده پایان‌نامه را به فارسی، وارد کنید.
% توجه داشته باشید که جدول حاوی مشخصات پروژه/پایان‌نامه/رساله و همچنین، مشخصات داخل آن، به طور خودکار، درج می‌شود.
%%%%%%%%%%%%%%%%%%%%%%%%%%%%%%%%%%%%
% دانشکده، آموزشکده و یا پژوهشکده  خود را وارد کنید
\faculty{دانشکده مهندسی کامپیوتر}
% گرایش و گروه آموزشی خود را وارد کنید
\department{}
% عنوان پایان‌نامه را وارد کنید
\fatitle{بررسی الگوریتم‌های هوش مصنوعی در پیش بینی مصرف انرژی ساختمان‌ها}
% نام استاد(ان) راهنما را وارد کنید
\firstsupervisor{دکتر رضا صفابخش}
% \secondsupervisor{}
% نام استاد(دان) مشاور را وارد کنید. چنانچه استاد مشاور ندارید، دستور پایین را غیرفعال کنید.
%\firstadvisor{نام کامل استاد مشاور}
%\secondadvisor{استاد مشاور دوم}
% نام نویسنده را وارد کنید
\name{فرشید }
% نام خانوادگی نویسنده را وارد کنید
\surname{نوشی}
%%%%%%%%%%%%%%%%%%%%%%%%%%%%%%%%%%
\thesisdate{فروردین ۱۴۰۱}

% چکیده پایان‌نامه را وارد کنید
\fa-abstract{
    پیش‌بینی مصرف انرژی برای ساختمان‌ها ارزش بسیار زیادی در تحقیقات بهره‌وری انرژی و پایداری دارد. مدل‌های پیش‌بینی دقیق انرژی پیامدهای متعددی برای برنامه‌ریزی و بهینه‌سازی انرژی ساختمان‌ها و پردیس‌ها دارد. برای ساختمان های جدید، 
    که در آن داده های ثبت شده گذشته در دسترس نیست، از روش های شبیه سازی کامپیوتری برای تجزیه و تحلیل انرژی و پیش بینی سناریوهای آینده استفاده می شود. 
    با این حال، برای ساختمان‌های موجود با داده‌های انرژی سری زمانی ثبت‌شده تاریخی، تکنیک‌های آماری و یادگیری ماشینی دقیق‌تر و سریع‌تر ثابت شده‌اند. این مطالعه مروری جامع از تکنیک‌های یادگیری ماشین موجود برای پیش‌بینی مصرف انرژی سری زمانی ارائه می‌کند.
     اگرچه تاکید بر یک تجزیه و تحلیل داده های سری زمانی منفرد است، اما بررسی فقط به آن محدود نمی شود زیرا داده های انرژی اغلب با سایر متغیرهای سری زمانی مانند آب و هوای بیرون و شرایط محیطی داخلی تجزیه و تحلیل می شوند. 
     نه روش محبوب پیش‌بینی که بر اساس پلتفرم یادگیری ماشینی است، تجزیه و تحلیل می‌شوند. یک بررسی و تحلیل عمیق از "مدل ترکیبی"، که ترکیبی از دو یا چند تکنیک پیش‌بینی است نیز ارائه شده است. ترکیبات مختلف مدل ترکیبی موثرترین در پیش‌بینی انرژی سری زمانی برای ساختمان هستند.
}

% کلمات کلیدی پایان‌نامه را وارد کنید
\keywords{یادگیری ماشین، هوش مصنوعی،‌ پیشبینی داده های سری زمانی‌، مصرف انرژی ساختمان‌ها‌}



\AUTtitle
%%%%%%%%%%%%%%%%%%%%%%%%%%%%%%%%%%
\vspace*{7cm}
\thispagestyle{empty}
\begin{center}
\includegraphics[height=5cm,width=12cm]{besm}
\end{center}