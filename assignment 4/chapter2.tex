\chapter{روش‌های پیش‌بینی مصرف انرژی ساختمان‌ها}
پیش‌بینی داده‌های سری زمانی از زمان گذشته مورد توجه محققین و متخصصین بوده است. در نتیجه در گذر زمان روش‌های متنوعی برای این موضوع پیشنهاد شده‌اند.
.آمار و احتمالات از علوم بسیار قدیمی بشریت محسوب میشود یکی از روش‌های قدیمی برای پیش‌بینی داده‌های سری زمانی میباشد در کنار این روش، روش‌های مهندسی نیز در دهه‌های گذشته استفاده شده‌اند.
در این فصل برای ارائه‌ی یک دید جامع و مناسب در مورد پیش‌بینی مصرف انرژی ساختمان‌ها در مورد هر دو روش گفته شده صحبت میکنیم و در نهایت در مورد روش‌های هوش‌مصنوعی توضیحاتی را ارائه میکنیم.


\section{روش آماری}

مدل‌های رگرسیون آماری صرفاً مصرف انرژی یا شاخص انرژی را با متغیرهای تأثیرگذار مرتبط می‌کنند. این مدل‌های تجربی از داده‌های عملکرد تاریخی ایجاد شده‌اند، به این معنی که قبل از آموزش مدل‌ها، باید داده‌های تاریخی کافی را جمع‌آوری کنیم. تحقیقات زیادی بر روی مدل های رگرسیون در مورد مسائل زیر انجام شده است. اولین مورد پیش بینی مصرف انرژی بر روی متغیرهای ساده شده مانند یک یا چند پارامتر آب و هوا است. مورد دوم پیش بینی برخی از شاخص انرژی مفید است. مورد سوم، تخمین پارامترهای مهم مصرف انرژی، مانند ضریب تلفات حرارتی کل، ظرفیت حرارتی کل و ضریب افزایش است که در تحلیل رفتار حرارتی ساختمان یا سیستم‌های سطح فرعی مفید هستند.
\\
در برخی از مدل‌های مهندسی ساده‌شده، از رگرسیون برای ارتباط مصرف انرژی با متغیرهای آب و هوایی برای به دست آوردن امضای انرژی استفاده می‌شود \cite{pfafferott2005thermal,bauer1998simplified}. بائر\footnote{\lr{Bauer}} و اسکارتزینی\footnote{\lr{Scartezzini}} \cite{bauer1998simplified} یک روش رگرسیون را برای انجام محاسبات گرمایش و سرمایش به طور همزمان با پرداختن به سودهای داخلی و همچنین خورشیدی پیشنهاد کردند. دار\footnote{\lr{Dhar}} و همکاران \cite{dhar1998modeling,dhar1999fourier} بار گرمایش و سرمایش را در ساختمان‌های تجاری با دمای حباب خشک در فضای باز به عنوان تنها متغیر آب و هوا مدل‌سازی کرد. یک مدل سری فوریه مبتنی بر دما برای نشان دادن وابستگی غیرخطی بارهای گرمایش و سرمایش به زمان و دما پیشنهاد شد. اگر داده‌های رطوبت و خورشید نیز در دسترس باشد، آنها استفاده از مدل سری فوریه تعمیم‌یافته را پیشنهاد کردند زیرا ارتباط مهندسی بیشتر و توانایی پیش‌بینی بالاتری دارد. همچنین با در نظر گرفتن دمای حباب خشک به عنوان متغیر واحد برای توسعه مدل، لی\footnote{\lr{Lei}} و هو\footnote{\lr{Hu}} \cite{lei2009baseline} مدل‌های رگرسیونی را برای پیش‌بینی صرفه‌جویی در انرژی از پروژه‌های مقاوم‌سازی ساختمان‌های اداری در یک منطقه تابستانی گرم و زمستانی سرد ارزیابی کردند. آنها نشان دادند که یک مدل خطی تک متغیری برای مدلسازی مصرف انرژی در شرایط آب و هوایی گرم و سرد کافی و کاربردی است. ما\footnote{\lr{Ma}} و همکاران\cite{ma2010study} روش‌های رگرسیون خطی چندگانه و خود رگرسیون را برای پیش‌بینی مصرف انرژی ماهانه برای ساختمان‌های عمومی در مقیاس بزرگ ادغام کرد. در کار چو\footnote{\lr{Cho}} و همکاران.\cite{cho2004effect}، مدل رگرسیون در اندازه‌گیری‌های 1 روزه، 1 هفته‌ای و 3 ماهه ایجاد شد که منجر به خطای پیش‌بینی در مصرف انرژی سالانه 100\%، 30\%، 6\% شد. این نتایج نشان می دهد که طول دوره اندازه گیری به شدت بر مدل های رگرسیون وابسته به دما تأثیر می گذارد.
\\
در مورد پیش‌بینی شاخص انرژی، لام\footnote{\lr{Lam}} و همکاران.\cite{lam2010principal} از تجزیه و تحلیل اجزای اصلی \footnote{\lr{PCA}} برای ایجاد یک شاخص آب و هوایی زد\footnote{\lr{Z}} با توجه به تابش خورشیدی جهانی، دمای حباب خشک و مرطوب استفاده کرد. آنها دریافتند که زد\footnotemark[10] همان روندی را دارد که بار سرمایشی شبیه سازی شده، تهویه مطبوع و مصرف انرژی ساختمان را نشان می دهد. این روند از تحلیل همبستگی با تحلیل رگرسیون خطی به دست آمد. این مدل بر اساس داده های 1979 تا 2007 توسعه یافته است.

\section{روش مهندسی}


روش های مهندسی از اصول فیزیکی برای محاسبه دینامیک حرارتی و رفتار انرژی در کل سطح ساختمان یا برای اجزای سطح فرعی استفاده می کنند. آنها در طول پنجاه سال گذشته به اندازه کافی توسعه یافته اند. این روش ها را می توان به طور تقریبی به دو دسته روش جامع تفصیلی و روش ساده شده طبقه بندی کرد. روش‌های جامع از توابع فیزیکی بسیار دقیق یا دینامیک حرارتی برای محاسبه دقیق، گام به گام، مصرف انرژی برای همه اجزای ساختمان با اطلاعات ساختمان و محیط‌زیست، مانند شرایط اقلیمی خارجی، ساخت و ساز ساختمان، بهره‌برداری، برنامه نرخ بهره‌برداری استفاده می‌کنند. و تجهیزات گرمایش و تهویه هوا به عنوان ورودی. در این مقاله، ما بر دیدگاه جهانی مدل‌ها و برنامه‌ها تمرکز می‌کنیم، در حالی که جزئیات این فرآیندهای محاسباتی بسیار فراتر از هدف این بررسی است. خوانندگان ممکن است برای جزئیات محاسبه به \cite{clarke2001energy} مراجعه کنند. برای سیستم های گرمایش و تهویه هوا، به طور خاص، محاسبه دقیق انرژی در \cite{mcquiston2004heating} معرفی شده است. سازمان بین المللی استاندارد سازی، استانداردی برای محاسبه مصرف انرژی برای گرمایش و سرمایش فضا برای یک ساختمان و اجزای آن ایجاد کرده است. صدها ابزار نرم افزاری برای ارزیابی کارایی انرژی، انرژی های تجدیدپذیر و پایداری در ساختمان ها توسعه یافته اند. برخی از آنها به طور گسترده برای توسعه استانداردهای انرژی ساختمان و تجزیه و تحلیل مصرف انرژی و اقدامات حفاظتی ساختمان ها مورد استفاده قرار گرفته اند. این ابزارها در مقاله های \cite{CRAWLEY2008661,al2001computer} بررسی شده اند. وزارت انرژی ایالات متحده فهرستی از تقریباً تمام ابزارهای شبیه سازی را که دائماً به روز می شود، نگهداری می کند.
\\
اگرچه این ابزارهای شبیه سازی دقیق موثر و دقیق هستند، اما در عمل مشکلاتی وجود دارد. از آنجایی که این ابزارها مبتنی بر اصول فیزیکی هستند، برای رسیدن به یک شبیه‌سازی دقیق، به جزئیات ساختمان و پارامترهای محیطی به عنوان داده‌های ورودی نیاز دارند. از یک طرف، این پارامترها برای بسیاری از سازمان ها در دسترس نیستند، به عنوان مثال، اطلاعات مربوط به هر اتاق در یک ساختمان بزرگ همیشه دشوار است. این عدم وجود ورودی های دقیق منجر به شبیه سازی با دقت پایین می شود. از سوی دیگر، به کارگیری این ابزارها معمولاً نیازمند کار کارشناسی خسته کننده است که انجام آن را دشوار و هزینه را کم می کند. به این دلایل برخی از محققان مدل های ساده تری را برای ارائه جایگزین هایی برای کاربردهای خاص پیشنهاد کرده اند.
\\
الحمود\footnote{\lr{Al-Homoud}} \cite{al2001computer} دو روش ساده شده را بررسی کرد. یکی روش درجه روز است که در آن تنها یک شاخص یعنی درجه روز تحلیل می شود. این روش حالت پایدار برای تخمین مصرف انرژی ساختمان های کوچک که در آن انرژی مبتنی بر پوشش غالب است، مناسب است. یکی دیگر از سطل، همچنین به عنوان روش فرکانس دما شناخته می شود، که می تواند برای مدل سازی ساختمان های بزرگ استفاده شود که در آن بارهای تولید شده داخلی غالب هستند یا بارها به طور خطی به اختلاف دمای بیرون و داخل خانه وابسته نیستند.
\\
شرایط آب و هوایی عوامل مهمی برای تعیین میزان مصرف انرژی ساختمان هستند. اینها اشکال مختلفی مانند دما، رطوبت، تابش خورشیدی، سرعت باد دارند و در طول زمان تغییر می کنند. مطالعات خاصی برای ساده کردن شرایط آب و هوایی در محاسبات انرژی ساختمان انجام شده است.
% \\
% وایت\footnote{\lr{White}} و ریچموت\footnote{\lr{Reichmuth}} [14] سعی کردند از میانگین دمای ماهانه برای پیش بینی مصرف انرژی ماهانه ساختمان استفاده کنند. این پیش‌بینی دقیق‌تر از روش‌های استاندارد است که معمولاً از گرمایش و سرمایش درجه روز یا سطل‌های دما استفاده می‌کنند. وستفال\footnote{\lr{Westphal}} و لامبرتس\footnote{\lr{Lamberts}} [15] بار گرمایش و سرمایش سالانه ساختمانهای غیر مسکونی را صرفاً بر اساس برخی متغیرهای آب و هوا، از جمله میانگین ماهانه حداکثر و حداقل دما، فشار اتمسفر، پوشش ابر و رطوبت نسبی پیش بینی کردند. نتایج آنها در مقایسه با ابزارهای شبیه سازی دقیق، دقت خوبی را در ساختمان های پوششی کم جرم نشان داد.
% \\
% علاوه بر شرایط آب و هوایی، ویژگی ساختمان نیز یکی دیگر از ویژگی های آن است
% عامل مهم و در عین حال پیچیده در تعیین عملکرد انرژی.
% کالیبراسیون یکی دیگر از موضوعات مهم در شبیه سازی انرژی ساختمان است. با تنظیم دقیق ورودی ها، می تواند رفتار انرژی شبیه سازی شده را دقیقاً با رفتار یک ساختمان خاص در واقعیت مطابقت دهد. پان\footnote{\lr{Pan}} و همکاران [19] شبیه سازی کالیبره شده را به عنوان یکی از روش های تجزیه و تحلیل انرژی ساختمان خلاصه کرد و آن را برای تجزیه و تحلیل مصرف انرژی یک ساختمان تجاری بلندمرتبه به کار برد. پس از مراحل کالیبراسیون مکرر، این مدل انرژی دقت بالایی در پیش‌بینی مصرف واقعی انرژی ساختمان مشخص شده نشان داد. بررسی دقیق شبیه سازی کالیبراسیون در [20] ارائه شده است. از آنجایی که کالیبراسیون یک کار خسته کننده و زمان بر است، می توان مشاهده کرد که انجام شبیه سازی دقیق با یک روش مهندسی دقیق از پیچیدگی بالایی برخوردار است.






\section{روش هوش مصنوعی}

روش های هوش مصنوعی در سال‌های اخیر رشد در زمینه‌ی پیش‌بینی مصرف انرژی ساختمان‌ها رشد بسیار زیادی داشته اند. به علت سهولت بهتر و دقت بالای این روش در سال‌های اخیر روش غالب در پیش‌بینی مصرف انرژی ساختمان‌ها این نوع روش‌ها بوده‌اند. 
در این بخش دو زیر بخش مهم از روش‌های هوش مصنوعی مورد بررسی قرار گرفته‌اند. 

\subsection{شبکه‌های عصبی\footnote{\lr{Neural Networks}}}

شبکه های عصبی مصنوعی پرکاربردترین مدل های هوش مصنوعی در کاربرد پیش بینی انرژی ساختمان هستند. این نوع مدل در حل مسائل غیر خطی خوب است و یک رویکرد موثر برای این کاربرد پیچیده است. در بیست سال گذشته، محققان از شبکه‌های عصبی مصنوعی برای تجزیه و تحلیل انواع مختلف مصرف انرژی ساختمان در شرایط مختلف، مانند بار گرمایش/سرمایش، مصرف برق، عملکرد و بهینه‌سازی اجزای سطح زیرین، تخمین پارامترهای مصرف استفاده کرده‌اند. در این بخش، مطالعات قبلی را مرور می کنیم و با توجه به کاربردهایی که به آن پرداخته شده، آنها را در گروه هایی قرار می دهیم. علاوه بر این، بهینه سازی مدل، مانند پیش فرآیند داده های ورودی و مقایسه بین شبکه های عصبی مصنوعی و سایر مدل ها، در پایان برجسته شده است. در سال 2006، کالوگیرو\footnote{\lr{Kalogirou}} \cite{kalogirou1997building} مروری کوتاه بر شبکه‌های عصبی مصنوعی در کاربردهای انرژی در ساختمان‌ها، از جمله سیستم‌های گرمایش آب خورشیدی، تابش خورشیدی، سرعت باد، توزیع جریان هوا در داخل اتاق، پیش‌بینی مصرف انرژی، دمای هوای داخل ساختمان و سیستم گرمایش و تهویه هوا انجام داد.
کالوگیرو\footnotemark[18] و همکاران \cite{kalogirou2006artificial} از شبکه های عصبی پس انتشار\footnote{\lr{back propagation neural networks}} برای پیش بینی بار گرمایش مورد نیاز ساختمان ها استفاده کرد. این مدل بر روی داده های مصرف 225 ساختمان آموزش داده شد که تا حد زیادی از فضاهای کوچک تا اتاق های بزرگ متفاوت است. اولوفسون و همکاران \cite{olofsson1998method} تقاضای گرمایش سالانه تعدادی از ساختمان‌های کوچک خانواده‌ای در شمال سوئد را پیش‌بینی کرد. بعداً، اولوفسون\footnote{\lr{Olofsson}} و اندرسون\footnote{\lr{Anderson}}\cite{olofsson2001long} یک شبکه عصبی ایجاد کردند که تقاضای انرژی بلندمدت (تقاضای گرمایش سالانه) را بر اساس داده‌های اندازه‌گیری شده کوتاه‌مدت (معمولاً 2 تا 5 هفته) با نرخ پیش‌بینی بالا برای ساختمان‌های تک خانواده پیش‌بینی می‌کند.


\subsection{ماشین بردار پشتیبان\footnote{\lr{Support Vector Machine(SVM)}}}

ماشین‌های بردار پشتیبان به طور فزاینده ای در تحقیقات و صنعت مورد استفاده قرار می گیرند. آنها مدل های بسیار موثری در حل مسائل غیر خطی حتی با مقادیر کم داده های آموزشی هستند. مطالعات بسیاری از این مدل ها در مورد تجزیه و تحلیل انرژی ساختمان در پنج سال گذشته انجام شده است.
دونگ\footnote{\lr{Dong}} و همکاران \cite{dong2005applying} برای اولین بار از ماشین بردار پشتیبان برای پیش بینی مصرف برق ماهانه چهار ساختمان در منطقه گرمسیری استفاده کرد. داده های سه ساله آموزش داده شد و مدل مشتق شده برای پیش بینی سودمندی مالک در آن سال بر روی داده های یک ساله اعمال شد. نتایج نشان دهنده عملکرد خوب ماشین‌های بردار پشتیبان در این مشکل بود.
\\
لای\footnote{\lr{Lai}} و همکاران\cite{lai2008vapnik} این مدل را بر مصرف برق یکساله یک ساختمان اعمال کرد. متغیرها شامل تغییرات آب و هوایی است. در آزمایشات آنها، این مدل از عملکرد یک سال استخراج شد و سپس بر روی رفتار سه ماهه آزمایش شد. آنها همچنین مدل را بر روی هر مجموعه داده روزانه آزمایش کردند تا پایداری این رویکرد را در دوره‌های کوتاه تأیید کنند. علاوه بر این، آنها اغتشاش را به صورت دستی به بخش خاصی از عملکرد تاریخی اضافه کردند و از این مدل برای تشخیص اغتشاش با بررسی تغییر وزن های کمک کننده استفاده کردند.
\\
لیانگ\footnote{\lr{Liang}} و دو\footnote{\lr{Du}} \cite{liang2007model} یک روش تشخیص عیب مقرون به صرفه را برای سیستم های گرمایش و تهویه هوا با ترکیب مدل فیزیکی و یک ماشین بردار پشتیبان ارائه کردند. با استفاده از طبقه‌بندی کننده چهار لایه ماشین بردار پشتیبان، می‌توان وضعیت عادی و سه خطای احتمالی را با تعداد کمی از نمونه‌های آموزشی به سرعت و با دقت تشخیص داد.


\section{خلاصه}

با توجه به توصیف و تحلیل فوق، بدیهی است که برای ارزیابی سیستم انرژی ساختمان، از سطح زیرسیستم تا سطح ساختمان و حتی سطح منطقه ای یا ملی، محاسبات زیادی مورد نیاز است. هر مدل در موارد خاصی از کاربردها مزایای خاص خود را دارد. مدل مهندسی تغییرات زیادی را نشان می دهد. ملاحظات زیادی می تواند در توسعه این مدل دخیل باشد. می تواند یک مدل بسیار پیچیده و جامع باشد که برای محاسبات دقیق قابل استفاده است. در مقابل، با اتخاذ برخی استراتژی‌های ساده‌کننده، می‌توان آن را به یک مدل سبک تبدیل کرد و با حفظ دقت، توسعه آن آسان است. یک اشکال رایج پذیرفته شده این مدل مهندسی دقیق این است که به دلیل پیچیدگی زیاد و کمبود اطلاعات ورودی، اجرای آن در عمل دشوار است. توسعه مدل آماری نسبتاً آسان است اما اشکالات آن نیز مشهود است که عبارتند از عدم دقت و عدم انعطاف پذیری. شبکه‌های عصبی مصنوعی و ماشین‌های بردار پشتیبان در حل مسائل غیر خطی خوب هستند و آنها را برای پیش بینی انرژی ساختمان کاربردی می کند. تا زمانی که انتخاب مدل و تنظیم پارامترها به خوبی انجام شود، آنها می توانند پیش بینی بسیار دقیقی ارائه دهند. ماشین‌های بردار پشتیبان حتی در بسیاری از موارد عملکرد بهتری نسبت به شبکه‌های عصبی مصنوعی نشان می دهند\cite{li2010prediction}. معایب این دو نوع مدل این است که به داده های عملکرد تاریخی کافی نیاز دارند و بسیار پیچیده هستند. تجزیه و تحلیل مقایسه ای این مدل های رایج در جدول 1 خلاصه شده است.
\begin{table}
    \begin{tabular}{ |p{2cm}|p{2cm}|p{2cm}|p{2cm}|p{2cm}|p{2cm}|  }
        \hline
        متد & پیچیدگی مدل  &سادگی استفاده&سرعت اجرا& نیازهای ورودی& دقت\\
        \hline
        مهندسی دقیق& نسبتاً بالا& غیرساده& کم& با جزئیات& نسبتاً بالا\\
        \hline
        مهندسی ساده‌سازی شده& بالا& ساده& بالا& ساده‌سازی شده& بالا\\
        \hline
        آماری& معمولی& ساده& نسبتاً بالا& داده‌های تاریخی& معمولی\\
        \hline
        شبکه‌های عصبی مصنوعی& بالا& غیرساده& بالا& داده‌های تاریخی& بالا\\
        \hline
        ماشین بردار پشتیبان& نسبتاً بالا& غیرساده& کم& داده‌های تاریخی& نسبتاً بالا\\
        \hline
        \end{tabular}
        \caption[خلاصه‌ی کلی از متد‌های پیش‌بینی و ویژگی های هریک]{خلاصه‌ی کلی از متد‌های پیش‌بینی و ویژگی های هریک}
\end{table}