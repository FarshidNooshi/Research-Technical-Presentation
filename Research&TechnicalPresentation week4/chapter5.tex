\chapter{مشخصات یک پایان نامه و گزارش علمی}

اگرچه براي همه انواع نوشته‌ها، مشخصات و ويژگي‌هاي واحد و معيني نمي‌توان ذكر كرد، با اين حال در یک پایان نامه یا گزارش علمی باید نکات و موارد کلی که در این فصل ذکر می‌شود، بطور کامل رعایت شده باشد. 

دقت كنيد كه پس از عنوان فصل بايد حداقل توضیحی کوتاه در مورد موضوع نوشته شود و نمي‌توان مستقيماً بعد از آن عنوان بخش را نوشت و همين طور پس از عناوين بخش‌ها و زيربخش‌ها.(مانند دستورالعمل حاضر)
\section{برخورداری از غنای علمی }

يك پایان نامه بايد پیش از هر چيز به‌لحاظ علمي از غناي لازم برخوردار باشد. يعني هدف و پيام روشني داشته باشد و از پيش‌زمينه علمي، بيان دلايل علمی، ارجاعات مورد نیاز و نتيجه‌گيري شفاف بهره ببرد. 

\section{ارجاع به‌موقع و صحیح به منابع دیگر}
هر جمله‌ای که در یک پایان نامه نوشته می‌شود یا یک جمله کاملاً بدیهی است یا باید دلیل آن بیان شود و یا اینکه باید به منبعی که آن موضوع را نقل یا اثبات کرده، ارجاع داده شود. اگر مطلب يا گفتاري از منبعی عيناً در گزارش نقل مي‌شود، بايد آن مطلب داخل گيومه قرار گيرد و با ذكر ماخذ و شماره صفحه، به آن اشاره گردد.


\section{ساده‌نویسی }
سادگی از ضروريات يك نوشته است. نويسنده بايد ساده، روان و در عين حال شيوا و رسا بنويسد و عبارات مبهم، جملات پيچيده و كلمات نامأنوس در نوشته خود به‌كار نبرد. اگر چه افراط در اين امر نيز، به شيوايي نوشته صدمه مي‌زند. به‌كارگیری لغات و اصطلاحات دشوار و دور از ذهن و عبارات و جملات نامنظم و مبهم موجب ايجاد اشكال در فهم خواننده خواهد شد‌. 

 براي ساده‌نويسي بايد در حد امكان از به‌كارگيری كلمات «مي‌بايست»، «بايستي»، «گرديد»، «بوده باشد» و مانند آنها كه تكلف‌آور، غلط مصطلح و يا غيرشيوا هستند، به‌جای «بايد»، «است»، «شد» و مثل آنها، اجتناب شود‌.‌ همين‌طور، «در‌جهت» نمی‌تواند جايگزين خوبی برای كلمه روانی مثل «برای» باشد‌. ‌كلمات و جملات روان و ساده مي‌توانند اغلب مفاهيم را براحتی منتقل كنند‌.‌
 
دقت در تنظیم بندها (پاراگراف‌ها) نيز كمك شاياني به روانی و سادگی فهم مطلب مي‌كند‌.‌ بندهای طولانی نيز مانند جملات طولانی مي‌توانند خسته‌كننده باشند و خواننده را سردرگم كنند‌.‌ يك بند نبايد کمتر از سه یا چهار سطر یا بيشتر از $10$ تا 15 سطر باشد.‌ 

\section{وحدت موضوع}

نویسنده بايد در سراسر نوشته از اصل موضوع دور نيافتد و تمام بحث‌ها، مثال‌ها و اجزاي نوشته با هماهنگي كامل، پيرامون موضوع اصلي باشد و تاثيري واحد در ذهن خواننده القا كند. 
\section{اختصار}

پایان نامه یا گزارش علمی بايد در حد امكان، مختصر و مفيد باشد و از بحث‌هاي غير ضروري در آن پرهيز شود. نوشتن مطالب ارزشمندي كه هيچ ربطي به موضوع ندارد، فاقد ارزش علمي است.
\section{رعایت نكات دستوري و نشانه‌گذاري}
در سراسر پایان نامه بايد قواعد دستوري رعايت شود و اركان و اجزاي جمله در جاي مناسب خود آورده شود. همچنین رعايت قواعد نشانه‌گذاري سبب مي‌شود كه بيان نويسنده روشن باشد و خواننده به سهولت و با کمترین صرف انرژی مطالب را مطالعه و درك كند.
\section{توجه به معلومات ذهنی مخاطب}
نويسنده بايد همواره مخاطب خود را در برابر خود تصور كند و با توجه به معلومات ذهني مخاطب  تمامي پیش‌نیازهای لازم براي درك مطالب مورد بحث را، از پیش براي مخاطب فراهم كند.

\section{رعایت مراحل اصولی نگارش}
هر کار علمی زمانی به بهترین شکل قابل انجام است که بر اساس یک برنامه‌ریزی مشخص انجام شود. تهیه یک متن علمي با کیفیت نیز نیازمند برنامه‌ریزی مناسب و اجرای منظم آن می‌باشد. مراحل نگارش را عموماً می‌توان به ترتیب زیر درنظر گرفت:
\begin{itemize}


\item	تهيه فهرستی از عناوین اصلي و فرعی که باید نوشته شود
\item 	اولویت‌بندی و تعیین ترتیب منطقی فصل‌ها و بخش‌های گزارش
\item 	گردآوري اطلاعات اولیه راجع به هر بخش و زیربخش
\item 	تدوین مطالب جدیدی که باید به قلم نگارنده به گزارش اضافه شود
\item 	تایپ كردن مطالب با رعایت کامل نکاتی که در این دستورالعمل آموزش داده می‌شود
\end{itemize}
رعایت نظم و ترتیب در اجرای مراحل ذکر شده هم فرآیند تهیه پایان نامه یا گزارش علمی را برای نگارنده آسان می‌کند و هم کیفیت نگارش را به میزان قابل توجهی افزایش می‌دهد.