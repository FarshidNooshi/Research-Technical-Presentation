\chapter{ارزیابی نتایج تجربی برروی مجموعه‌های داده}
%\thispagestyle{empty}

پس از معرفی الگوریتم‌های متنوع و شرح مسئله نوبت به بررسی و ارزیابی آن‌ها برروی مجموعه‌های داده‌ متنوع میشود. در این فصل ابتدا معیار‌های ارزیابی معرفی شده اند و سپس 
نتایج الگوریتم‌های مختلف معرفی شده برای پیش‌بینی داده‌های سری‌های زمانی مصرف انرژی ساختمان‌ها معرفی و مورد بحث قرار گرفته اند.
و در نهایت بهینه ترین روش‌ها در صورت ممکن معرفی شدند.
\section{معیار‌های ارزیابی}

میانگین درصد مطلق خطا \LTRfootnote{{mean absolute percentage error (MAPE)}}، همچنین به عنوان میانگین درصد انحراف مطلق \LTRfootnote{{mean absolute percentage deviation (MAPD)}}
 شناخته می شود، معیاری برای دقت پیش بینی یک روش پیش بینی در آمار است. معمولاً دقت را به عنوان نسبتی که با فرمول \ref{eq:mape} تعریف می شود بیان می کند:

 \begin{equation}\label{eq:mape}
    MAPE = \frac{100\%}{n}\sum_{t=1}^{n}\left |\frac{A_t - F_t}{A_t}\right|
\end{equation}
\noindent
که در آن $A_t$ مقدار واقعی و $F_t$ ارزش پیش بینی شده است. تفاوت آنها بر ارزش واقعی $A_t$ تقسیم می شود. مقدار مطلق در این نسبت برای هر نقطه پیش بینی شده
 در زمان جمع می شود
 و بر تعداد نقاط برازش $n$ تقسیم می شود.

 
 \noindent
 میانگین درصد مطلق خطا معمولاً به عنوان یک تابع ضرر\LTRfootnote{{Loss function}} برای مسائل رگرسیونی و در ارزیابی مدل استفاده می‌شود،
 زیرا تفسیر بسیار شهودی آن بر حسب خطای نسبی است.
 \\
 در قسمت‌های بعدی این بخش به روز خواهد شد و دو معیار دیگر نیز به آن اضافه خواهند شد.

\section{نتیجه ی روش های مورد بررسی}
در این بخش با توجه به دانش کسب شده از بخش‌های قبل 
مانند توضیحات انواع الگوریتم‌ها و نحوه‌ی کارکردن هرکدام و معرفی معیار‌های ارزیابی‌مان الگوریتم‌های مختلف  برروی دیتاست‌های متنوع اجرا شده اند.
در جدول زیر یک خلاصه از روش‌های معرفی شده آمده اند.

تکنیک‌های پیش‌بینی سری‌های زمانی فوق با موفقیت برای پیش‌بینی مصرف انرژی ساختمان استفاده شده‌اند. هر تکنیک دارای ویژگی های سودمند خاصی است که باید به طور مناسب در مورد مورد استفاده قرار گیرد. 
این بخش مزایا و معایب هر یک را برای تکنیک هایی که قبلاً مورد بحث قرار گرفت، روشن می کند.

\begin{table}\label{tb:result}
    \begin{tabular}{ |p{2cm}|p{2cm}|p{2cm}|p{2.4cm}|p{2cm}|p{2cm}|  }
        \hline
        مدل & نوع دیتا  & طول داده‌ی تمرین & میانگین درصد مطلق خطا($\%$) & زمان صرف شده& ارجاع\\
        \hline
        شبکه‌های عصبی مصنوعی& بار برق ساعتی& یک سال & ۸۱.۱-۶۹.۱& ۳۵ ثانیه& \cite{chitsaz2015short}\\
        \hline
        ماشین بردار پشتیبان& داده های بار خنک کننده ساعتی و آب و هوا & یک ماه& ۰۱۶.۱-۰۰۱.۱& کمتر از یک دقیقه& \cite{li2009applying}\\
        \hline
        سری فازی زمانی & بار برق روزانه & شش ماه& ۶۳.۱-۲۳.۱& خارج از دسترس& \cite{efendi2015new}\\
        \hline
        میانگین متحرک خودهمبسته یکپارچه & پیک تقاضای برق ماهانه & شش سال& ۵۹.۲-۰۵.۱ & خارج از دسترس& \cite{rallapalli2012forecasting}\\
        \hline
        روش‌های ترکیبی & شرح داده خواهند شد & -& - & - & -\\
        \hline
        \end{tabular}
        \caption[نتیجه‌ی الگوریتم‌های هوش‌‌ مصنوعی]{نتیجه‌ی الگوریتم‌های هوش‌‌ مصنوعی}
    \end{table}
    

\section{مقایسه و بررسی روش‌های معرفی شده}
شبکه‌های عصبی مصنوعی\LTRfootnote{{Artificial Neural Network (ANN)}} مزایای بیشتری نسبت به مدل‌های آماری دارد، زیرا می‌تواند نقشه‌برداری را انجام دهد
رابطه ورودی و خروجی بدون ایجاد وابستگی پیچیده
در میان ورودی ها مشاهده می شود که شبکه‌های عصبی مصنوعی عملکرد بسیار بهتری را در مقایسه با تکنیک های قبلی
 اجرا شده برای نقشه برداری غیر خطی ارائه می دهد. شبکه‌های عصبی مصنوعی می تواند مدلسازی غیرخطی را بدون هیچ گونه دانش قبلی در مورد روابط بین ورودی و خروجی انجام دهد
متغیرها بنابراین، اینها مدل‌سازی عمومی‌تر و انعطاف‌پذیرتر هستند
تکنیک برای پیش بینی با این حال، شبکه‌های عصبی مصنوعی وابسته به مقداردهی اولیه وزن است، حداقل های محلی و مشکل همگرایی کند را نشان می دهد. علاوه بر این،
 به دست آوردن تعادل
  بین بیش‌برازش و تعمیم برای
   شبکه‌های عصبی مصنوعی همیشه یک چالش است. از سوی دیگر، میانگین متحرک خودهمبسته یکپارچه\LTRfootnote{{AutoRegressive Integrated Moving Average (ARIMA)}} یک تقریب جهانی است که به اندازه کافی عناصر رگرسیون و میانگین گرفته شده است  به طوری که می توان تقریب
  را متناسب با هر سری زمانی
  انجام داد. با این حال، شناسایی میانگین متحرک خودهمبسته یکپارچه پیچیده و زمان‌بر است و بسیاری از مدل‌های میانگین متحرک خودهمبسته یکپارچه هیچ تفسیر ساختاری ندارند. شناسایی و تخمین می‌تواند
   به‌دلیل تأثیرات پرت به شدت تحریف شود.
   برای مشکلات پیش‌بینی انرژی ساختمان، ماشین بردار پشتیبان\LTRfootnote{{Support Vector Machine (SVM)}} می‌تواند با مسائلی مانند نمونه کوچک، غیرخطی، ابعاد بالا و حداقل نقاط محلی مقابله کند. 
   معمولاً داده‌های بلندمدت دارای ویژگی‌های نمونه‌های کوچک هستند و این امر ماشین بردار پشتیبان را برای پیش‌بینی داده‌های بلندمدت مناسب می‌کند.
    علاوه بر این، روش رگرسیون ماشین بردار پشتیبان نیز مانند شبکه‌های عصبی مصنوعی قابلیت مناسبی برای برازش و تعمیم دارد. یک نقطه قوت خاص استفاده از یک تابع هسته برای معرفی غیرخطی بودن
     و مقابله با داده های ساختار یافته دلخواه است. 
    با این حال، درست مانند دو تکنیک دیگر ذکر شده در بالا، ماشین بردار پشتیبان فاقد شفافیت نتایج است و به راحتی قابل تفسیر نیست. معمولاً تابع هسته به پارامترهای خاصی بستگی دارد
 که برای دستیابی به نتایج خوب باید بهینه شوند.
 \\
 در مورد تکنیک استدلال مبتی بر مورد مزیت اصلی این است که تقریباً برای هر دامنه ای قابل اعمال است. سیستم استدلال مبتی بر مورد\LTRfootnote{{Case-based Reasoning (CBR)}} سعی نمی کند قوانینی را بین پارامترهای مسئله پیدا کند،
  بلکه فقط سعی می کند مسائل مشابه را در داده ها بیابد 
 و از راه حل های این مسائل
  به عنوان راه حلی برای مورد مورد مطالعه استفاده کند. 
 مزیت دوم این است که رویکرد استدلال مبتی بر مورد به یادگیری و حل مسئله بسیار شبیه است
فرآیندهای شناختی انسان استدلال مبتی بر مورد ارتباط بین رویدادهای مشابه در گذشته و آینده را تشخیص می دهد. با این حال، استدلال مبتی بر مورد به ندرت برای پیش بینی سری های زمانی استفاده می شود.
 دلیل این امر این است که استفاده از استدلال مبتی بر مورد برای پردازش سری های زمانی جنبه های جدیدی را معرفی می کند که توالی ها می توانند بسیار طولانی و با طول های مختلف باشند. 
 علاوه بر این، حجم عظیم داده و وجود نویز در این داده‌ها مرتبط با محدودیت‌های زمان واقعی، سیستم استدلال مبتی بر مورد را غیرضروری می‌کند.
  در مقابل، تکنیک‌های پیش‌بینی فازی در حل عدم قطعیت‌ها در پیش‌بینی بار بسیار خوب هستند.
 با این حال، الگوهای زمانی توسط مناطق سفت و سخت تعریف می شوند که تنظیم آنها در هنگام وجود نویز در مجموعه داده دشوار است. 
 اغلب از پیچیدگی محاسباتی بالایی برخوردار است و ثبات ندارد.
  برای تکنیک پیش‌بینی فازی، سری‌های زمانی باید به سری‌های ثابت و تناوبی تبدیل شوند تا الگوهایی در سری‌های زمانی استخراج شود.
  \\
  مدل های ترکیبی برای مقابله با مشکلات دنیای واقعی که
   اغلب ماهیت پیچیده ای دارند، مناسب هستند. یک مدل یادگیری ماشینی ممکن است نتواند پیچیدگی‌های ایجاد انرژی و داده‌های عملیاتی را به تصویر بکشد.
   در چنین مواردی، استفاده از مدل هیبریدی می تواند سودمند باشد. روش‌های ترکیبی ابزارهای تحلیلی قوی برای دسته بزرگی از مسائل پیچیده هستند
   که متمایل به روش‌های کلاسیک سنتی نیستند. با ترکیب روش‌های مختلف، ساختارهای خودهمبستگی پیچیده در داده‌ها را می‌توان با دقت بیشتری مدل‌سازی کرد. 
  علاوه بر این، مشکل انتخاب مدل را می توان با کمی تلاش بیشتر آسان کرد. برای مثال یکی از مهم ترین و پرکاربردترین مدل های سری زمانی مدل شبکه‌های عصبی مصنوعی-میانگین متحرک خودهمبسته یکپارچه
  \LTRfootnote{{Artificial Neural Network AutoRegressive Integrated Moving Average (ANNARIMA)}} است. این مدل به عنوان ترکیبی 
  از میانگین متحرک خودهمبسته یکپارچه و شبکه‌های عصبی مصنوعی از قدرت منحصر به فرد شبکه‌های عصبی مصنوعی و میانگین متحرک خودهمبسته یکپارچه به ترتیب در مدل‌سازی غیرخطی و خطی بهره می‌برد. به نظر می رسد که مزایای چنین روش هایی در برخورد
   با سری های غیر ثابت قابل توجه باشد. مولفه غیر خطی غیر ثابت را می توان با استفاده از مدل شبکه‌های عصبی مصنوعی و مولفه خطی ثابت
    و باقیمانده را می توان با مدل میانگین متحرک خودهمبسته یکپارچه مدل کرد. به روشی مشابه، روش های دیگر را می توان برای بهبود دقت پیش بینی ترکیب کرد


    در پایان این بخش یک جدول بزرگ و جامع در مورد مقایسه‌ی الگوریتم‌های مختلف و مزایا و معایب هرکدام خواهد آمد.
\section{خلاصه}


در این فصل به ارزیابی و مقایسه‌ی الگوریتم‌های معرفی شده در فصل سوم کردیم. ابتدا معیار ارزیابی مناسببی برای مقایسه‌ی الگوریتم‌های پیش‌بینی داده‌های سری‌های زمانی
به نام میانگین درصد مطلق خطا
معرفی کردیم و در ادامه با استفاده از مراجع گزارشمان نتیجه‌ی اجرای هریک از الگوریتم‌های معرفی شده برروی مجموعه‌ی داده‌های خود مراجع گزارش داده شدند. 
در بخش سوم این فصل به مقایسه ی مفصل الگوریتم‌های خود پرداختیم و سعی در مقایسه‌ی آن‌ها با همدیگر کردیم. 
نکته‌ی مهمی که در بررسی‌ها متوجه آن شدیم این است که ترکیب چندین مدل هوش مصنوعی و یادگیری ماشین نتیجه‌ی مطلوبتری را
نسبت به یک مدل منفرد فراهم آورد. 

