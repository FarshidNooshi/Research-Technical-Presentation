% دانشکده، آموزشکده و یا پژوهشکده  خود را وارد کنید
\faculty{دانشکده مهندسی کامپیوتر}
% گرایش و گروه آموزشی خود را وارد کنید
\department{گزارش نوشتاری}
% عنوان پایان‌نامه را وارد کنید
\fatitle{بررسی الگوریتم‌های هوش مصنوعی در پیش بینی مصرف انرژی ساختمان‌ها}
% نام استاد(ان) راهنما را وارد کنید
\firstsupervisor{دکتر رضا صفابخش}
\secondsupervisor{}
% نام استاد(دان) مشاور را وارد کنید. چنانچه استاد مشاور ندارید، دستور پایین را غیرفعال کنید.
%\firstadvisor{نام کامل استاد مشاور}
%\secondadvisor{استاد مشاور دوم}
% نام نویسنده را وارد کنید
\name{فرشید }
% نام خانوادگی نویسنده را وارد کنید
\surname{نوشی}
%%%%%%%%%%%%%%%%%%%%%%%%%%%%%%%%%%
\thesisdate{فروردین ۱۴۰۱}

% چکیده پایان‌نامه را وارد کنید
\fa-abstract{سرطان چشم با رشدی روزافزون بینایی جامعه را تهدید می‌کند. اگرچه چنین بیماری‌هایی به موقع درمان نشوند خطرناک و لاعلاج به‌نظر می‌رسند، اما با تشخیص سریع و زودهنگامشان می‌توان از شدت رنج تحمیل شده بر افراد به حد چشمگیری کاست. یشگیری از بیماری و یا تشخیص زودهنگام آن به ویژه زماني که بیماری هنوز علامتي ایجاد نکرده است و فرد احساس ناخوشی ندارد، مهم تر و آسان تر از درمان بیماری در مراحل پیشرفته و توأم با ظهور عوارض است.
\\
هوش مصنوعی و مخصوصا زمینه‌ی تشخیص ناهنجاری یکی از ابزارهایی است که با پیشرفتهای اخیر در این رشته مهندسی رایانه، می‌تواند بسیار کمک‌کننده باشد. در این پژوهش قصد داریم با خصوصیات این روش تشخیص آشنا شده و دستاوردهایش را بررسی کنیم.
}


% کلمات کلیدی پایان‌نامه را وارد کنید
\keywords{بیماری‌های چشمی، یادگیری ماشین، پردازش تصویر، تشخیص ناهنجاری، سرطان چشم}



\AUTtitle
%%%%%%%%%%%%%%%%%%%%%%%%%%%%%%%%%%
\vspace*{7cm}
\thispagestyle{empty}
\begin{center}
\includegraphics[height=5cm,width=12cm]{besm}
\end{center}