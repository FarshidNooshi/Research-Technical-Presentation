\chapter{مقدمه}

آژانس بین‌المللی انرژی، بهره‌وری انرژی در ساختمان‌ها را به عنوان یکی از پنج اقدام برای تضمین کربن زدایی طولانی مدت بخش انرژی شناسایی کرده است\cite{DEB2017902}
در کنار مزایای زیست محیطی، بهره وری انرژی ساختمان دارای مزایای اقتصادی گسترده ای نیز می باشد. ساختمان هایی با سیستم های انرژی کارآمد و استراتژی های مدیریتی هزینه های عملیاتی بسیار کمتری دارند. اکنون بسیاری از کشورها اجرای قوانین و مقررات انرژی را برای انواع ساختمان ها تسریع کرده اند. این مقررات الزامات اساسی برای دستیابی به یک طراحی کارآمد انرژی برای ساختمان‌های جدید با هدف کاهش مصرف انرژی نهایی و انتشار \lr{CO2} مرتبط را ترسیم می‌کند. علاوه بر این، بسیاری از نرم افزارهای کامپیوتری نیز برای طراحی بهینه انرژی ساختمان های جدید توسعه یافته و به طور گسترده پیاده سازی شده اند. برخی از محبوب ترین آنها \lr{EnergyPlus, DOE-2, eQUEST, IES, ECOTECT} و غیره هستند. مطالعه دقیقی در مورد تکنیک های موجود تجزیه و تحلیل انرژی ساختمان به کمک کامپیوتر و ابزارهای نرم افزاری در [2،3] موجود است. این مقررات و ابزارهای کامپیوتری مربوط به ساختمان های جدید است و در واقع بسیار موثر هستند. با این حال، هنگامی که ساختمان عملکردی دارد، عوامل زیادی بر رفتار انرژی یک ساختمان حاکم است، مانند شرایط آب و هوایی، برنامه اشغال، خواص حرارتی مصالح ساختمانی، فعل و انفعالات پیچیده سیستم‌های انرژی مانند \lr{HVAC} و روشنایی و غیره. به دلیل این فعل و انفعالات پیچیده، دقیق محاسبه مصرف انرژی از طریق مدل سازی شبیه سازی کامپیوتری بسیار دشوار است. به این دلایل، تکنیک های داده محور برای تجزیه و تحلیل انرژی ساختمان ساختمان های موجود بسیار حیاتی است. این تکنیک‌ها بر داده‌های ثبت‌شده گذشته تکیه دارند و تلاش می‌کنند مصرف انرژی را بر اساس الگوهای مصرف انرژی قبلی مدل‌سازی کنند. سایر عوامل مؤثر بر مصرف انرژی را می توان برای بهبود دقت چنین مدل های سری زمانی استفاده کرد. این تکنیک‌ها که از داده‌های گذشته استفاده می‌کنند، اغلب تحت «یادگیری ماشینی» قرار می‌گیرند و در دو دهه اخیر به طور فعال در مطالعات پیش‌بینی انرژی ساختمان به کار رفته‌اند. مزایا و معایب چنین تکنیک‌های مبتنی بر داده در جدول 1 ارائه شده است. جزئیات تکنیک ها در بخش های بعدی توضیح داده شده است.