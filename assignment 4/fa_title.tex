%% -!TEX root = AUTthesis.tex
% در این فایل، عنوان پایان‌نامه، مشخصات خود، متن تقدیمی‌، ستایش، سپاس‌گزاری و چکیده پایان‌نامه را به فارسی، وارد کنید.
% توجه داشته باشید که جدول حاوی مشخصات پروژه/پایان‌نامه/رساله و همچنین، مشخصات داخل آن، به طور خودکار، درج می‌شود.
%%%%%%%%%%%%%%%%%%%%%%%%%%%%%%%%%%%%
% دانشکده، آموزشکده و یا پژوهشکده  خود را وارد کنید
\faculty{دانشکده مهندسی کامپیوتر}
% گرایش و گروه آموزشی خود را وارد کنید
\department{}
% عنوان پایان‌نامه را وارد کنید
\fatitle{بررسی الگوریتم‌های هوش مصنوعی در پیش‌بینی مصرف انرژی ساختمان‌ها}
% نام استاد(ان) راهنما را وارد کنید
\firstsupervisor{دکتر رضا صفابخش}
% \secondsupervisor{}
% نام استاد(دان) مشاور را وارد کنید. چنانچه استاد مشاور ندارید، دستور پایین را غیرفعال کنید.
%\firstadvisor{نام کامل استاد مشاور}
%\secondadvisor{استاد مشاور دوم}
% نام نویسنده را وارد کنید
\name{فرشید }
% نام خانوادگی نویسنده را وارد کنید
\surname{نوشی}
%%%%%%%%%%%%%%%%%%%%%%%%%%%%%%%%%%
\thesisdate{اردیبهشت ۱۴۰۱}

% چکیده پایان‌نامه را وارد کنید
\fa-abstract{
    پیش‌بینی مصرف انرژی برای ساختمان‌ها ارزش بسیار زیادی در تحقیقات بهره‌وری انرژی و پایداری دارد. مدل‌های پیش‌بینی دقیق انرژی، فواید متعددی در برنامه‌ریزی
     و بهینه‌سازی انرژی ساختمان‌ها و پردیس‌ها دارند. برای ساختمان‌های جدید، 
    که در آن داده‌های ثبت شده گذشته در دسترس نیستند، از روش‌های شبیه‌سازی کامپیوتری برای تجزیه و تحلیل انرژی و پیش‌بینی سناریوهای آینده استفاده می‌شود.
    با این‌ حال برای ساختمان‌های موجود با داده‌های انرژی سری زمانی ثبت‌شده گذشته، تکنیک‌های آماری و یادگیری ماشین دقیق‌تر و سریع‌تر عمل کرده‌اند. 
    این گزارش بررسی‌ای بر الگوریتم‌های هوش مصنوعی موجود برای پیش‌بینی مصرف انرژی سری زمانی انجام داده است.
     اگرچه تاکید بر یک تجزیه‌و‌تحلیل داده‌های سری زمانی منفرد است، اما بررسی فقط به آن محدود نمی‌شود زیرا داده‌های انرژی 
     اغلب با سایر متغیرهای سری زمانی مانند آب‌و‌هوای بیرون و شرایط محیطی داخلی براساس روش محبوب پیش‌بینی که براساس یادگیری ماشین می‌باشد، تجزیه و تحلیل می‌شوند 
      یک بررسی از "مدل ترکیبی"، که ترکیبی از دو یا چند الگوریتم پیش‌بینی است نیز ارائه شده است.
      ترکیبات مختلف مدل ترکیبی موثرترین الگوریتم در پیش‌بینی انرژی سری زمانی برای ساختمان‌ها هستند.
}

% کلمات کلیدی پایان‌نامه را وارد کنید
\keywords{یادگیری ماشین، هوش مصنوعی،‌ پیش‌بینی داده‌های سری زمانی‌، مصرف انرژی ساختمان‌ها‌}



\AUTtitle{a}
%%%%%%%%%%%%%%%%%%%%%%%%%%%%%%%%%%
\vspace*{7cm}
\thispagestyle{empty}
\begin{center}

\end{center}