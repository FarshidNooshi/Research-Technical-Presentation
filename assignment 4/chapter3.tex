\chapter{ارزیابی نتایج تجربی برروی مجموعه‌های داده}
%\thispagestyle{empty}

پس از معرفی الگوریتم‌های متنوع و شرح مسئله نوبت به بررسی و ارزیابی آن‌ها برروی مجموعه‌های داده‌ متنوع میشود. در این فصل ابتدا معیار‌های ارزیابی معرفی شده اند و سپس 
نتایج الگوریتم‌های مختلف معرفی شده برای پیش‌بینی داده‌های سری‌های زمانی مصرف انرژی ساختمان‌ها معرفی و مورد بحث قرار گرفته اند.
و در نهایت بهینه ترین روش‌ها در صورت ممکن معرفی شدند.
\section{معیار‌های ارزیابی}

میانگین درصد مطلق خطا \footnote{\lr{mean absolute percentage error (MAPE)}}، همچنین به عنوان میانگین درصد انحراف مطلق \footnote{\lr{mean absolute percentage deviation (MAPD)}}
 شناخته می شود، معیاری برای دقت پیش بینی یک روش پیش بینی در آمار است. معمولاً دقت را به عنوان نسبتی که با فرمول \ref{eq:mape} تعریف می شود بیان می کند:

 \begin{equation}\label{eq:mape}
    MAPE = \frac{100\%}{n}\sum_{t=1}^{n}\left |\frac{A_t - F_t}{A_t}\right|
\end{equation}
\noindent
که در آن $A_t$ مقدار واقعی و $F_t$ ارزش پیش بینی شده است. تفاوت آنها بر ارزش واقعی $A_t$ تقسیم می شود. مقدار مطلق در این نسبت برای هر نقطه پیش بینی شده
 در زمان جمع می شود
 و بر تعداد نقاط برازش $n$ تقسیم می شود.

\noindent
 میانگین درصد مطلق خطا\footnotemark[1] معمولاً به عنوان یک تابع ضرر\footnote{\lr{Loss function}} برای مسائل رگرسیونی و در ارزیابی مدل استفاده می‌شود،
  زیرا تفسیر بسیار شهودی آن بر حسب خطای نسبی است.

\section{مقایسه ی روش های مورد بررسی}
\section{انتخاب بهینه ترین روش پیشنهادی}
\section{خلاصه}
